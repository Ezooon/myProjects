\documentclass{article}

\usepackage{tabularx}
\usepackage{booktabs}
\usepackage[normalem]{ulem}
\usepackage{xcolor}

\title{SE 3XA3: Problem Statement\\BlockBuilder}

\author{Team 28, OAC
		\\ Andrew Lucentini, lucenta
		\\ Owen McNeil, mcneilo
		\\ Christopher DiBussolo, dibussoc
}

\date{December 2, 2018}

%\input{../Comments}

\begin{document}

\begin{table}[hp]
\caption{Revision History} \label{TblRevisionHistory}
\begin{tabularx}{\textwidth}{llX}
\toprule
\textbf{Date} & \textbf{Developer(s)} & \textbf{Change}\\
\midrule
September 21, 2018 & 
\begin{tabular}{@{}c@{}}Owen McNeil \\ Andrew Lucentini \\ Chris DiBussolo\end{tabular}
 & Creation of problem statement document\\
\midrule
\textcolor{red}{December 4, 2018} & 
\begin{tabular}{@{}c@{}} \textcolor{red}{Owen McNeil} \\ \textcolor{red}{Andrew Lucentini} \\ \textcolor{red}{Chris DiBussolo}\end{tabular}
& \textcolor{red}{Changes made for Rev1, highlighted in red}\\
\bottomrule
\end{tabularx}
\end{table}

\newpage

\maketitle

\section*{Problem Statement}
\subsection*{Introduction}

Video games are arguably one of the most current popular forms of entertainment. A well known category of video games are open world sandbox games. These types of games allow users to freely create and modify the game world around them. Sandbox games bring users the joy of playing video games while encouraging creative thought. We want to create an open world sandbox game that users will be able to play for free with ease of access.

\subsection*{Importance}

One of the most popular open world sandbox games is Minecraft. A major drawback to Minecraft is that users must pay money to play the game. This is where BlockBuilder will come into play. BlockBuilder is based off an open source simplified version of Minecraft. BlockBuilder will be an free, quick, and easily learnable open world sandbox game for people to unleash their creativity in a fun and engaging way. The game will capture the basic principles of Minecraft so users can focus solely on creating. Users will be able to modify the in game landscape to create anything their imagination desires quickly and without spending a dime. In addition, we want to re-implement this simplified open source version of Minecraft with the addition of creating extensive software documentation. Software documentation enhances readability, the ability to improve correctness, and will make it easier for this software to be further developed.

\subsection*{Context}
BlockBuilder will be run in a desktop \textcolor{red}{and/or laptop }environment. \textcolor{red}{The software will most likely be used in a house environment, but may occasionally be used anywhere that a user wishes to use their laptop. }The software can be installed in a few steps and ran with the click of a mouse. BlockBuilder can be played by people of all ages. There are very little requirements to play BlockBuilder as it is a simplified version of its inspiration, requiring only the installation of python and pyglet to run it. The ease of access of this version of Minecraft, particularly the fact that no purchase is necessary, makes it more appealing to kids and young adults who lack a credit card, which is the target demographic of Minecraft in the first place.
%\wss{comment}

%\ds{comment}

%\mj{comment}

%\cm{comment}

%\mh{comment}

\end{document}