\documentclass[12pt]{article}

\usepackage{graphicx}
\usepackage{paralist}
\usepackage{amsfonts}
\usepackage{amsmath}
\usepackage{hhline}
\usepackage{booktabs}
\usepackage{multirow}
\usepackage{multicol}
\usepackage{tabularx}
\usepackage[normalem]{ulem}
\usepackage{xcolor}

\oddsidemargin 0mm
\evensidemargin 0mm
\textwidth 160mm
\textheight 200mm
\renewcommand\baselinestretch{1.0}

\pagestyle {plain}
\pagenumbering{arabic}

\newcounter{stepnum}

\usepackage{color}

\newif\ifcomments\commentstrue

\ifcomments
\newcommand{\authornote}[3]{\textcolor{#1}{[#3 ---#2]}}
\newcommand{\todo}[1]{\textcolor{red}{[TODO: #1]}}
\else
\newcommand{\authornote}[3]{}
\newcommand{\todo}[1]{}
\fi

\newcommand{\wss}[1]{\authornote{blue}{SS}{#1}}

\title{SE 3XA3: Module Interface Specification\\BlockBuilder}

\author{Team 28, OAC
		\\ Owen McNeil, mcneilo
		\\ Christopher DiBussolo, dibussoc
		\\ Andrew Lucentini, lucenta
}
\begin {document}
 
\maketitle

This
document shows the complete specification for the modules used for running BlockBuilder.

\begin{table}[hp]
\caption{Revision History} \label{TblRevisionHistory}
\begin{tabularx}{\textwidth}{llX}
\toprule
\textbf{Date} & \textbf{Developer(s)} & \textbf{Change}\\
\midrule
November 9, 2018 & All members & Pushed Rev0 to repo\\
November 2, 2018 & All members & Creation of Rev 0\\
\textcolor{red}{December 1, 2018} & \textcolor{red}{All members} & \textcolor{red}{Begin creating changes for rev1}\\
\textcolor{red}{December 5, 2018} & \textcolor{red}{All members} & \textcolor{red}{Rev1 complete}\\
\bottomrule
\end{tabularx}
\end{table}

\newpage

\section* {Function Module}

\subsection*{Module}

Function

\subsection* {Uses}

N/A

\subsection* {Syntax}

\subsubsection* {Exported Constants}

\textcolor{red}{SECTOR\_SIZE = 16}

\subsubsection* {Exported Types}

None

\subsubsection* {Exported Access Programs}

\begin{tabular}{| l | l | l | l |}
\hline
\textbf{Routine name} & \textbf{In} & \textbf{Out} & \textbf{Exceptions}\\
\hline
normalize &$\mathbb{R}, \mathbb{R}, \mathbb{R}$ & Set of $\mathbb{R}$ & \sout{None} \textcolor{red}{ invalid\_size}\\
\hline
sector & Set of $\mathbb{R}$ & Set of $\mathbb{R}$ & invalid\_size\\
\hline
\end{tabular}

\subsection* {Semantics}

\subsubsection* {State Variables}

None

\subsubsection* {Environment Variables}

None

\subsubsection* {State Invariant}

None

\subsubsection* {Assumptions}

\begin{itemize}
\item The mathematical operator $\backslash$ represents integer division. For example $ 8 \backslash 5 = 1$.
\end{itemize}

\subsubsection* {Access Routine Semantics}

\noindent normalize(\sout{$x, y, z$} \textcolor{red}{ $ position$}):
\begin{itemize}
\item transition: None
\item output: \sout{$out := \{ round(x), round(y), round(z) \}$}\\
\textcolor{red}{$out := \{ round(position[0]), round(position[1]), round(position[2]) \}$}
\item exception: \sout{None}\\
\textcolor{red}{$ exc := (|position| \neq 3) \Rightarrow invalid\_size $}
\end{itemize}

\noindent sectorize($position$):
\begin{itemize}
\item transition: None
\item output: \sout{$out := normalize(position[0] \backslash SECTOR\_SIZE, 0,
position[2] \backslash SECTOR\_SIZE  )$}\\
\textcolor{red}{$out := (normalize(position)[0] \backslash SECTOR\_SIZE, 0,
normalize(position)[2] \backslash SECTOR\_SIZE  )$}
\item exception: $ exc := (|position| \neq 3) \Rightarrow invalid\_size $\\
\end{itemize}

\subsection*{Local Functions}

round: $\mathbb{R} \rightarrow \mathbb{Z}$

\noindent round$(x) \equiv$
\medskip
$ \mbox{The value of the real x is rounded to the nearest integer.}$\\
\medskip
\newpage

\section* {Block Module}

\subsection*{Module}

Block

\subsection* {Uses}

N/A

\subsection* {Syntax}

\subsubsection* {Exported Constants}
\sout{SECTOR\_SIZE = 16}
\textcolor{red}{None}


\subsubsection* {Exported Types}

GRASS = allFacesCoordinates((1, 0), (0, 1), (0, 0)) \\
SAND = allFacesCoordinates((1, 1), (1, 1), (1, 1)) \\
BRICK = allFacesCoordinates((2, 0), (2, 0), (2, 0)) \\
STONE = allFacesCoordinates((2, 1), (2, 1), (2, 1))\\
\textcolor{red}{DIRT = allFacesCoordinates((0, 1), (0, 1), (0, 1))}\\
inventoryT = \{GRASS, SAND, BRICK, STONE\}\\

\subsubsection* {Exported Access Programs}

\begin{tabular}{| l | l | l | l |}
\hline
\textbf{Routine name} & \textbf{In} & \textbf{Out} & \textbf{Exceptions}\\
\hline
cubeVertices & $\mathbb{R}, \mathbb{R}, \mathbb{R}, \mathbb{R}_{>0}$ & Set of $\mathbb{R}$ & nNotPos\\
\hline
textureCoordinate & Set of $\mathbb{R}$ & Set of $\mathbb{R}$ & invalid\_size\\
\hline
allFacesCoordinates & Set of $\mathbb{R}$, Set of $\mathbb{R}$, Set of $\mathbb{R}$ & Set of $\mathbb{R}$ & invalid\_size\\
\hline
\end{tabular}

\subsection* {Semantics}

\subsubsection* {State Variables}

None

\subsubsection* {Environment Variables}

None

\subsubsection* {State Invariant}

None

\subsubsection* {Assumptions}

None

\subsubsection* {Access Routine Semantics}

cubeVertices($x, y, z, n$):
\begin{itemize}
\item transition: None
\item output: $\mathit{out} := \{ x-n,y+n,z-n, x-n,y+n,z+n, x+n,y+n,z+n, x+n,y+n,z-n, x-n,y-n,z-n, x+n,y-n,z-n, x+n,y-n,z+n, x-n,y-n,z+n, x-n,y-n,z-n, x-n,y-n,z+n, x-n,y+n,z+n, x-n,y+n,z-n, x+n,y-n,z+n, x+n,y-n,z-n, x+n,y+n,z-n, x+n,y+n,z+n, x-n,y-n,z+n, x+n,y-n,z+n, x+n,y+n,z+n, x-n,y+n,z+n, x+n,y-n,z-n, x-n,y-n,z-n, x-n,y+n,z-n, x+n,y+n,z-n \}$
        
\item exception: $ (n < 0) \Rightarrow $ nNotPos\\
\end{itemize}

\noindent textureCoordinates($p$):
\begin{itemize}
\item transition: None
\item output: $out := \{ p[0] * \frac{1.0}{4}, p[1] * \frac{1.0}{4}, p[0] * \frac{1.0}{4} + \frac{1.0}{4}, p[1] * \frac{1.0}{4}, p[0] * \frac{1.0}{4} + \frac{1.0}{4}, p[1] * \frac{1.0}{4} + \frac{1.0}{4}, p[0] * \frac{1.0}{4}, p[1] * \frac{1.0}{4} + \frac{1.0}{4} \}$
\item exception: $ exc := (|p| \neq 2) \Rightarrow invalid\_size $\\
\end{itemize}

\noindent allFacesCoordinates($top, bottom, side$):
\begin{itemize}
\item transition: None
\item output: $out := \{ texCoord(top), texCoord(bottom), texCoord(side), texCoord(side),\\
texCoord(side), texCoord(side) \}$
\item exception: $ exc := (|top| \neq 2 \lor |bottom| \neq 2 \lor |side| \neq 2) \Rightarrow invalid\_size $\\
\end{itemize}

\newpage

\section* {Constants Module}

\subsection*{Module}

Constants

\subsection* {Uses}

None

\subsection* {Syntax}

\subsubsection* {Exported Constants}

TICKS\_PER\_SEC = 60\\
WALKING\_SPEED = 5\\
FLYING\_SPEED = 20\\
GRAVITY = 20.0\\
MAX\_JUMP\_HEIGHT = 1.0\\
JUMP\_SPEED = $\sqrt{\mbox{(2 * GRAVITY * MAX\_JUMP\_HEIGHT)}}$\\
TERMINAL\_VELOCITY = 50\\
PLAYER\_HEIGHT = 2\\
TEXTURE\_PATH = 'texture.png'

\subsubsection* {Exported Types}

None

\subsubsection* {Exported Access Programs}

None

\subsection* {Semantics}

\subsubsection* {State Variables}

None

\subsubsection* {State Invariant}

None

\newpage

\section* {World}

\subsection* {Module}

World

\subsection* {Uses}

Block, Constants, Function

\subsection* {Syntax}

\subsubsection* {Exported Types}

World = ?

\subsubsection* {Exported Constants}

None

\subsubsection* {Exported Access Programs}

\begin{tabular}{| l | l | l | p{3.5cm} |}
\hline
\textbf{Routine name} & \textbf{In} & \textbf{Out} & \textbf{Exceptions}\\
\hline
World & ~ & World & ~\\
\hline
GenerateWorld & ~ & ~ & ~\\
\hline
hitTest & Set of $\mathbb{R}$, Set of $\mathbb{R}$, $\mathbb{Z}$ & Set of $\mathbb{R}$ & invalid\_Distance\\
\hline
\textcolor{red}{exposed} & \textcolor{red}{Set of $\mathbb{R}$} & \textcolor{red}{$\mathbb{B}$} & \\
\hline
addBlock & Set of $\mathbb{R}$, \textcolor{red}{Set of $\mathbb{R}$}, inventoryT & ~ & ~ \\
\hline
removeBlock & Set of $\mathbb{R}$ & ~ & ~\\
\hline
showBlock & Set of $\mathbb{R}$ & ~ & \\
\hline
hideBlock & Set of $\mathbb{R}$ & ~ & \\
\hline
checkSurrounding & Set of $\mathbb{R}$  & ~ & \\
\hline
\textcolor{red}{showSector} & \textcolor{red}{Set of $\mathbb{R}$} & ~ & \\
\hline
\textcolor{red}{hideSector} & \textcolor{red}{Set of $\mathbb{R}$}  & ~ & \\
\hline
\textcolor{red}{changeSector} & \textcolor{red}{Set of $\mathbb{R}$, Set of $\mathbb{R}$}  & ~ & \\
\hline

\end{tabular}

\subsection* {Semantics}

\subsubsection* {State Variables}

blockSet: Set of ((\mbox{Set of}  $\mathbb{R}) \times inventoryT)$\\ \textit{\# A set representing all of the blocks in the world at a given position with a given inventoryT.}\\

\noindent shownBlocks: Set of ((\mbox{Set of}  $\mathbb{R}) \times inventoryT)$\\ \textit{\# A set representing all of the blocks in the world that are visable to the player at a given position with a given inventoryT.}\\

\noindent \textcolor{red}{sectors: Set of ((\mbox{Set of}  $\mathbb{R}) \times$ (\mbox{Set of}  $\mathbb{R}$))}\\ 
\textit{ \textcolor{red}{ \# Mapping from sector to a list of positions inside that sector.}}\\

\subsubsection* {Environment Variables}

None

\subsubsection* {State Invariant}

None

\subsubsection* {Assumptions}

\begin{itemize}
\item The constructor Window is called for each object instance before any other
access routine is called for that object.  The constructor cannot be called on
an existing object. 

\item The generateWorld() access routine is called after World() but before any other access routine.

\item The showBlock(position) access routine assumes the block at position has already been added to the world with addBlock.

\item All Set of $\mathbb{R}$ defined as any inputs or outputs to access routines have a length of 3.


\item The operator / represents set difference. I.e. s := s / x means the set s becomes s with the element x removed.


\end{itemize}

\subsubsection* {Access Routine Semantics}

World():
\begin{itemize}
\item transition: $blockSet, shownBlocks := \{\}, \{\} $
\item output: $\mathit{out} := \mathit{self}$
\item exception: None\\
\end{itemize}

\noindent generateWorld():
\begin{itemize}
\item transition: $blockSet := $ Set of randomly generated, life-like landforms using elements from inventoryT.
\item ouput := None
\item exception: None\\
\end{itemize}

\noindent hitTest(pos, vec, distance):
\begin{itemize}
\item transition: None
\item output: $out := $ \textit{A set of $\mathbb{R}$ representing the position of a block if it is intersected with the player's line of sight and vector, and is less than distance blocks away.}
\item exception: $exc := (distance < 0) \Rightarrow invalid\_Distance$\\
\end{itemize}

\noindent \textcolor{red}{exposed(position)}:
\begin{itemize}
\item \textcolor{red}{transition: None}
\item \textcolor{red}{output: $out := $ \textit{True if one of the faces from the block at position does not exist in the blockSet, otherwise false}} 
\item \textcolor{red}{exception: None}\\
\end{itemize}

\noindent addBlock(position, \textcolor{red}{playerPos, } texture):
\begin{itemize}
\item transition: $(< (sectorize(position), texture) > \not\in blockSet \textcolor{red}{\land position \neq playerPos}) \Rightarrow
blockSet := blockSet || < (sectorize(position), texture) >$
\item output: None
\item exception: None\\
\end{itemize}

\noindent removeBlock(position):
\begin{itemize}
\item transition: $(< (sectorize(position), texture) > \in blockSet)) \Rightarrow
blockSet := blockSet / < (sectorize(position), texture) >$
\item output: None
\item exception: None\\
\end{itemize}


\noindent showBlock(position):
\begin{itemize}
\item transition: (The block at position can be seen by the player) $\Rightarrow
(shownBlocks := shownBlocks|| < sectorize(position), getTextureFromSet(sectorize(position)) > $
\item output: None
\item exception: None\\
\end{itemize}

\noindent hideBlock(position):
\begin{itemize}
\item transition: (The block at position cannot be seen by the player) $\Rightarrow
(shownBlocks := shownBlocks / < sectorize(position), getTextureFromSet(sectorize(position)) > $
\item output: None
\item exception: None\\
\end{itemize}

\noindent checkSurrounding(position):
\begin{itemize}
\item transition: Check all blocks surrounding `position` and ensure their visual state is current. This means hiding blocks that are not exposed and
ensuring that all exposed blocks are shown. Usually used after a block
is added or removed. The routine will call showBlock and hideBlock accordingly.
\item output: None
\item exception: None\\
\end{itemize}

\noindent \textcolor{red}{showSector(sector)}:
\begin{itemize}
\item \textcolor{red}{transition: Ensure all blocks in the given sector that should be shown are drawn using addBlock.}
\item \textcolor{red}{output: None} 
\item \textcolor{red}{exception: None}\\
\end{itemize}

\noindent \textcolor{red}{hideSector(sector)}:
\begin{itemize}
\item \textcolor{red}{transition: Ensure all blocks in the given sector that should be shown are drawn using addBlock.}
\item \textcolor{red}{output: None} 
\item \textcolor{red}{exception: None}\\
\end{itemize}

\noindent \textcolor{red}{changeSector(before, after)}:
\begin{itemize}
\item \textcolor{red}{transition:  Move from sector before to sector after.}
\item \textcolor{red}{output: None} 
\item \textcolor{red}{exception: None}\\
\end{itemize}



\subsection*{Local Functions}

getTextureFromSet: $ \mbox{Set of } \mathbb{R} \rightarrow inventoryT$

\noindent getTextureFromSet$(p) \equiv$
\medskip
The texture in the set blockSet corresponding to the element with position equivalient to p.\\
\medskip
\newpage

\section* {Window Module}

\subsection*{Module}

Window

\subsection* {Uses}

World, Block, Constants, Function

\subsection* {Syntax}

\subsubsection* {Exported Types}

Window = ?

\subsubsection* {Exported Access Programs}

\begin{tabular}{| l | l | l | l |}
\hline
\textbf{Routine name} & \textbf{In} & \textbf{Out} & \textbf{Exceptions}\\
\hline
Window & ~ & Window & \\
\hline
setExclusiveMouse & $\mathbb{B}$ & ~ & ~\\
\hline
getSightVector & ~ & Set of $\mathbb{R}$ & ~\\
\hline
getMotionVector & ~ & Set of $\mathbb{R}$ & ~\\
\hline
Collision & Set of $\mathbb{R}, \mathbb{Z}$ & Set of $\mathbb{R}$ & ~\\
\hline
on\_mouse\_press & Keyboard Mouse click & ~ & ~\\
\hline
on\_mouse\_motion & $\mathbb{R}$, $\mathbb{R}$, $\mathbb{R}$, $\mathbb{R}$ & ~ & ~\\
\hline
on\_key\_press & keyInput & ~ & ~\\
\hline
on\_key\_release & keyInput & ~ & ~\\
\hline
draw & ~ & ~ & ~\\
\hline
\end{tabular}

\subsection* {Semantics}

\subsubsection* {State Variables}
Exclusive: $\mathbb{B}$\textit{ \#Determines if the mouse is captured by the window}\\
Flying: $\mathbb{B}$ \textit{ \#Determines if flying mode is on/off}\\
Strafe: Set of $\mathbb{Z}$ \textit{ \#Determines the direction of movement}\\
Position: Set of $\mathbb{R}$ \textit{ \#Defines the player's position in the world}\\
Rotation: Set of $\mathbb{R}$ \textit{ \#Defines the relative position of the screen}\\
Sector: Set of $\mathbb{Z}$ \textit{ \textcolor{red}{\#An integer list of sectors}}\\
Reticle: Generated Pyglet Graphics\\
dy: $\mathbb{R}$ \textit{ \#Defines the relative y velocity of the screen}\\
Inventory: \{GRASS, SAND, BRICK, STONE\} \textit{ \textcolor{red}{\#Set of blocks able to be placed by user}}\\
Block: inventoryT \textit{ \#The current block being used by the player}\\
World: World() \textit{ \textcolor{red}{\#A world object}}\\ 
Label: Generated Pyglet Label\\

\subsubsection* {Environment Variables}
keyInput: \{ "key.\_W", "key.\_S", "key.\_A", "key.\_D", "key.\_SPACE", "key.\_ESCAPE", "key.\_TAB", "key.\_1", "key.\_2", "key.\_3" \} \textit{ \#The set of keys corresponding to the keys on a keyboard with their respective names.} \\

\noindent leftClick : \textit{ \#A left click provided by a mouse/track pad}\\

\noindent rightClick: \textit{ \#A right click provided by a mouse/track pad}\\

\noindent cursorX : $\mathbb{R}$ \textit{ \#The speed at which the mouse is moving in the x direction (negative for left direction and positive for right direction}\\

\noindent cursorY: $\mathbb{R}$\textit{ \#The speed at which the mouse is moving in the y direction (negative for downward direction and positive for upward direction)}\\

\noindent TEXTURE\_PATH: \textit{ \#The path to the image used to load the textures.}\\

\subsubsection* {State Invariant}

$|Strafe| = 2 \land |Position| = 3 \land |Rotation| = 2 \land |Sector| = 3$\\

\subsubsection* {Assumptions}

\begin{itemize}
\item The constructor Window is called for each object instance before any other
access routine is called for that object.  The constructor cannot be called on
an existing object. 

\item All access routines except for Window() and setExclusiveMouse(excl) are called by pyglet library TICKS\_PER\_SEC times a second. The access routines on\_mouse\_press, on\_mouse\_motion, on\_key\_press, and on\_key\_release are required for the pyglet library to read user input.

\item It is assumed that a 3D envorinment is generated with the pyglet library when Window() is called. The window acts as the player point of view and has a position in the Window given by the set of 3 $\mathbb{R}$ written \{x, y, z\}.
\end{itemize}
 
\subsubsection* {Access Routine Semantics}

Window():
\begin{itemize}
\item output: $out$ := A window with a default size (defined by the pyglet library) is created on the computer.
\item transition: Exclusive, Flying, Strafe, Position, Rotation, Sector, Reticle, dy, Block := False, False, \{0 , 0\}, \{0, 0, 0\}, \{0, 0\}, None, None, 0, Inventory[0]
\item exception: None\\
\end{itemize}

\noindent setExclusiveMouse($excl$):
\begin{itemize}
\item output: None
\item transition: $Exclusive := excl$\\
\textit{i.e The mouser cursor disapears and the pyglet window has exclusive access to the mouse if excl is true}
\item exception: None\\
\end{itemize}

\noindent getSightVector():
\begin{itemize}
\item output: $out := \{ cos(\frac{(Rotation[0] - 90)* \pi}{180}) * cos(\frac{Rotation[1]*\pi}{180}) , sin(\frac{Rotation[1]*\pi}{180}), sin(\frac{(Rotation[0] - 90)* \pi}{180}) * cos(\frac{Rotation[1]*\pi}{180})  \}$ \textit{i.e get the world coordinates of where the camera is looking}
\item transition: None 
\item exception: None\\
\end{itemize}

\noindent getMotionVector():
\begin{itemize}
\item output: \textit{out := The current motion vector of the screen is outputted as a set of three $\mathbb{R}$ labelled \{x, y, z\}, where each element represents the camera velocity in the x, y and z directions respectively.}
\item transition: None
\item exception: None\\
\end{itemize}

\noindent Collision(position, height):
\begin{itemize}
\item output: None
\item transition: \textit{Position := Given the player position (\{x, y, z\}) and PLAYER\_HEIGHT height, a new \{x, y, z\} position is calculated after taking into account any collisions with blocks existing in the world. A player cannot move into the square space defined by a block.}
\item exception: None\\
\end{itemize}

\noindent on\_mouse\_press($button$):
\begin{itemize}
\item output: None
\item transition:

\begin{tabular}{|p{8cm}|l|}
\hhline{~|-|}
\multicolumn{1}{r|}{} & \multicolumn{1}{l|}{$World :=$}\\
\hhline{|-|-|}
$button = rightClick \land $
$World.hitTest(position, getSightVector())$
$\ne NULL \land Exclusive = True$ & $World.addBlock(getSightVector(), Block)$\\
\hhline{|-|-|}
$button = leftClick  \land $
$World.hitTest(position, getSightVector())$
$\ne NULL \land Exclusive = True$ & $World.removeBlock(getSightVector())$\\
\hhline{|-|-|}
$Exclusive = False \land (button = rightClick \lor
button = leftClick)$  & $setExclusiveMouse(True)$\\
\hhline{|-|-|}
\end{tabular}
\item exception: None \\
\end{itemize}

\noindent on\_mouse\_motion(x, y, dx, dy):
\begin{itemize}
\item output: None 
\item transition: $(Exclusive = True) \Rightarrow \\Rotation := (x + cursorX * 0.2, min(max(-90, y + cursorY*0.2), 90))$ \\
\textit{Note: x and y are the position of the mouse on the screen, if Exclusive = True, these values are the center of the window.}
\item exception: None\\
\end{itemize}

\noindent on\_key\_press($symbol$):
\begin{itemize}
\item output: None 
\item transition: 
\begin{tabular}{|p{3.5cm}|l|}
\hhline{~|-|}
\multicolumn{1}{r|}{} & \multicolumn{1}{l|}{$transistion$}\\
\hhline{|-|-|}
$symbol = key.\_W$ & $Strafe[0] := Strafe[0] - 1$\\
\hhline{|-|-|}
$symbol = key.\_S$ & $Strafe[0] := Strafe[0] + 1$\\
\hhline{|-|-|}
$symbol = key.\_A$ & $Strafe[0] := Strafe[1] - 1$\\
\hhline{|-|-|}
$symbol = key.\_D$ & $Strafe[0] := Strafe[1] + 1$\\
\hhline{|-|-|}
$symbol = key.\_SPACE$ & $(dy = 0 \land Flying = False) \Rightarrow
dy = JUMP\_SPEED$\\
\hhline{|-|-|}
$symbol = key.\_ESCAPE$ & $Exclusive := False$\\
\hhline{|-|-|}
$symbol = key.\_TAB$ & $Flying := \lnot Flying$\\
\hhline{|-|-|}
$symbol = key.\_1$ & $Block := GRASS$\\
\hhline{|-|-|}
$symbol = key.\_2$ & $Block := SAND$\\
\hhline{|-|-|}
$symbol = key.\_3$ & $Block := BRICK$\\
\hhline{|-|-|}
$symbol = key.\_4$ & $Block := STONE$\\
\hhline{|-|-|}
\end{tabular}
\item exception: None \\
\end{itemize}

\noindent on\_key\_release($symbol$):
\begin{itemize}
\item output: None 
\item transition: 
\begin{tabular}{|p{3.5cm}|l|}
\hhline{~|-|}
\multicolumn{1}{r|}{} & \multicolumn{1}{l|}{$strafe[0] :=$}\\
\hhline{|-|-|}
$symbol = key.\_W$ & $strafe[0] + 1$\\
\hhline{|-|-|}
$symbol = key.\_S$ & $strafe[0] - 1$\\
\hhline{|-|-|}
$symbol = key.\_A$ & $strafe[1] + 1$\\
\hhline{|-|-|}
$symbol = key.\_D$ & $strafe[1] - 1$\\
\hhline{|-|-|}
\end{tabular}
\item exception: None \\
\end{itemize}

\noindent draw():
\begin{itemize}
\item output: None
\item transition: \textit{Draw the rectile, designated labels, and all block textures provided by the coordinates defined in each element from inventoryT read from TEXTURE\_PATH. Only the faces of the squares defined in the shown variable in the World variable are drawn.}
\item exception: None\\
\end{itemize}

\newpage



\end {document}
